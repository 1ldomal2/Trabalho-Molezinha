\documentclass[titlepage]{article}
\usepackage{hyperref}
\usepackage[T1]{fontenc}
\usepackage{setspace}
\usepackage{indentfirst}
%\usepackage{sbc-template}
\usepackage[brazil]{babel}
%\usepackage[lati11]{inputenc}

\title{Help Hoteis Molezinha}
\date{11 de Dezembro de 2017}
\author{Lucas Mateus Fernandes \and Leandro Soulza Pinheiro}
	\begin{document}
	\maketitle
	\newpage
	\tableofcontents %%Sumário
	\newpage

				\section{Requisitos Minimos}
					O software "Hoteis\_Molezinha" foi desenvoldido para ambientes Linux, o seu uso em outros sistemas operacionais é instavel e por isso é desaconselhavel.
				\section{Primeira Utilização}
					Ao iniciar o software será criada uma pasta chamada Arquivos no mesmo diretorio do do executavel, dentro  da pasta Arquivos contem todos os arquivos de backup.\\
					Apenas ao iniciar o Aplicativo pela primeira vez será escolhida a forma de manipulação, podendo ser arquivos binarios com a extensão .bin ou arquivos texto com a extensão .txt. Contudo há outras opçoes que não foram implementadas nesta vesão.\\
					Dentro de "Arquivos/Configuracoes.txt" ficará salvo 1 caso esteja em Modo Texto e 2 para Modo Binário.\\
					Após a coniguração inicial é pedido ao usario para que digite o Código Nome de Usuario e Senha para que possa ser efetuado o login com as permissoes adequadas.\\
					Por padrão já estão cadastradas duas contas que não podem ser alteradas\\

				
					\begin{table}[h]
						\begin{center}
							\begin{tabular}{|c|c|}
								\hline
									Codigo:1&Codigo:666\\
									Nome de usuario:1&Nome de usuario:Da50\\
									Senha:1&Senha:Manoel\\
								\hline
							\end{tabular}
						\end{center}
					\end{table}
				

				Ao efetuar o login é mostrado a tela com 6 opções de menu, onde será pedido para digitar de 1 a 5, sendo elas:

				\begin{itemize}
					\item Registros
					\item Reserva
					\item Pesquisas
					\item FeedBack
					\item Importar/Exportar XML
				\end{itemize}

				No menu registro é possivel Ler, Criar, Editar e Apagar, mas antes é preciso informar a opção referente ao módulo que deve ser: Hotel, Hospedes, Acomodações, Categoria, Produtos, Fornecedores e Funcionários.\\

				\newpage
				\section{Registro}
				\subsection{Hotel}
				Para Criar um Hotel é preciso informar alguns dados essenciais sendo eles:\\
				\begin{table}[h]%%Tabela
					\begin{tabular}{|c|c|}
						\hline
						Nome Fatansia & Onde deve ser informado qualquer tipo de caracte exceto o e \\ & [espaço] com no maximo 80 caracteres.\\
						\hline
						Razao Social & Onde deve ser informado qualquer tipo de caractere exceto \\ & o [espaço] com no maximo 80 caracteres.\\
						\hline
						Inscricao Estadual & Onde deve ser informado qualquer tipo de caractere exceto \\ & o [espaço] com no maximo 80 caracteres.\\
						\hline
						CNPJ & Onde deve ser informado qualquer tipo de caractere exceto \\ & o [espaço] com no maximo 80 caracteres.\\
						\hline
						Logradouro & Onde deve ser informado qualquer tipo de caractere exceto \\ & o [espaço] com no maximo 80 caracteres.\\
						\hline
						Numero & Onde deve ser informado qualquer tipo de caractere exceto \\ & o [espaço] com no maximo 80 caracteres.\\
						\hline
						Bairro & Onde deve ser informado qualquer tipo de caractere exceto \\ & o [espaço] com no maximo 80 caracteres.\\
						\hline
						Cidade & Onde deve ser informado qualquer tipo de caractere exceto \\ & o [espaço] com no maximo 80 caracteres.\\
						\hline
						Telefone & Onde deve ser informado qualquer tipo de caractere exceto \\ & o [espaço] com no maximo 80 caracteres.\\
						\hline
						Email & Onde deve ser informado qualquer tipo de caractere exceto \\ & o [espaço] com no maximo 80 caracteres.\\
						\hline
						Dono Gerente & Onde deve ser informado qualquer tipo de caractere exceto \\ & o [espaço] com no maximo 80 caracteres.\\
						\hline
						Telefone Gerente & Onde deve ser informado qualquer tipo de caractere exceto \\ & o [espaço] com no maximo 80 caracteres.\\
						\hline
						Check in & Onde deve ser informado qualquer tipo de caractere exceto \\ & o [espaço] com no maximo 80 caracteres.\\
						\hline
						Check out & Onde deve ser informado qualquer tipo de caractere exceto \\ & o [espaço] com no maximo 80 caracteres.\\
						\hline
						Lucro & Onde deve ser informado apenas um numero inteiro.\\
						\hline
					\end{tabular}
				\end{table}

			

				Ao Editar um hotel o usuário deve informar o Código do hotel a ser editado, que pode ser visto no menu Ler,ao inserir um Código valido(considera-se valido um código já cadastrado que nao tenha nenhuma acomodação vinculada a ele) e confirma sua ação,feito isso é preciso preencher novamente todas as informações presentes no menu criar que são elas (Nome Fantasia, Razao Social, Inscricao Estadual, CNPJ, 	Logradouro, Numero, Bairro, Cidade, 	Telefone, 	Email, 	Dono Gerente, 	Telefone Gerente, 	Check in, 	Check out, 	Lucro).\\

				Ao Apagar um hotel o usuário deve informar o Código do hotel a ser apagado,e confirmar sua ação. \\


				\newpage
				\subsection{Hospede}
				Para Criar um Hospede é preciso informar alguns dados essenciais sendo eles:\\
				\begin{table}[h]%%Tabela
					\begin{tabular}{|c|c|}
						\hline
						Nome  &Onde deve ser informado qualquer tipo de caractere exceto  \\ & o [espaço] com no maximo 80 caracteres.\\
						\hline
						CPF &Onde deve ser informado qualquer tipo de caractere exceto  \\ & o [espaço] com no maximo 80 caracteres.\\
						\hline
						Logradouro &Onde deve ser informado qualquer tipo de caractere exceto  \\ & o [espaço] com no maximo 80 caracteres.\\
						\hline
						Numero &Onde deve ser informado qualquer tipo de caractere exceto  \\ & o [espaço] com no maximo 80 caracteres.\\
						\hline
						Bairro &Onde deve ser informado qualquer tipo de caractere exceto  \\ & o [espaço] com no maximo 80 caracteres.\\
						\hline
						Cidade &Onde deve ser informado qualquer tipo de caractere exceto  \\ & o [espaço] com no maximo 80 caracteres.\\
						\hline
						Telefone &Onde deve ser informado qualquer tipo de caractere exceto  \\ & o [espaço] com no maximo 80 caracteres.\\
						\hline
						Email &Onde deve ser informado qualquer tipo de caractere exceto  \\ & o [espaço] com no maximo 80 caracteres.\\
						\hline
						Sexo &Onde deve ser informado qualquer tipo de caractere exceto  \\ & o [espaço] com no maximo 80 caracteres.\\
						\hline
						Estado Civil &Onde deve ser informado qualquer tipo de caractere exceto  \\ & o [espaço] com no maximo 80 caracteres.\\
						\hline
						Data Nascimento &Onde deve ser informado qualquer tipo de caractere exceto  \\ & o [espaço] com no maximo 80 caracteres.\\
						\hline
					\end{tabular}
				\end{table}

				Ao Editar um hospede o usuário deve informar o Código do hospede a ser editado, que pode ser visto no menu Ler,ao inserir um Código valido(considera-se valido um código já cadastrado) e confirma sua ação,feito isso é preciso preencher novamente todas as informaçoes presentes no menu criar que são elas (Nome, CPF,Logradouro
				Numero, Bairro, Cidade,Telefone, Email, Sexo, Estado\_Civil, Data\_Nascimento).\\

				Ao Apagar um hospede o usuário deve informar o Código do hotel a ser apagado,e confirmar sua ação. \\


				\newpage
				\subsection{Acomodacoes}
				Para Criar um Acomodacoes é preciso ja ter um hotel e uma categoria cadastradas e informar alguns dados essenciais sendo eles:\\
				\begin{table}[h]%%Tabela
					\begin{tabular}{|c|c|}
					\hline
					Descrição & Onde deve ser informado qualquer tipo de caractere exceto o \\& [espaço] com no maximo 999 caracteres.\\
					\hline
					Televisao & Onde deve ser informado um numero inteiro referente a quantidade.\\
					\hline
					Ar Condicionado & Onde deve ser informado um numero inteiro referente a quantidade.\\
					\hline
					Frigobar & Onde deve ser informado um numero inteiro referente a quantidade.\\
					\hline
					Internet & Onde deve ser informado um numero inteiro referente a quantidade.\\
					\hline
					Banheira & Onde deve ser informado um numero inteiro referente a quantidade.\\
					\hline
					Código Categoria & Onde deve ser informado a uma categoria ja criada.\\
					\hline
					Código Hotel & Onde deve ser informado a uma categoria ja criada.\\
					\hline
					\end{tabular}
				\end{table}

				Ao Editar uma Acomodacoes o usuário deve informar o Código do Acomodacoes a ser editado, que pode ser visto no menu Ler,ao inserir um Código valido(considera-se valido um código já cadastrado) e confirma sua ação,feito isso é preciso preencher novamente todas as informações presentes no menu criar, que são elas (Descrição, Televisao, Ar Condicionado, Frigobar, Internet, Banheira, Código Categoria, Código Hotel).\\

				Ao Apagar um Acomodacoes o usuário deve informar o Código da Acomodacoes a ser apagado,e confirmar sua ação.\\


				\newpage
				\subsection{Código Categoria}
					Para Criar um Código Categoria é preciso informar alguns dados essenciais sendo eles:
				\begin{table}[h]%%Tabela
					\begin{tabular}{|c|c|}
					\hline
					Descrição  & Onde deve ser informado qualquer tipo de \\& caractere exceto o [espaço] com no maximo 999 caracteres.\\
					\hline
					Valor Diaria  & Onde deve ser informado um numero flutuante \\& referente ao valor diaria.\\
					\hline
					Capacidade Adulto  & Onde deve ser informado um numero inteiro \\& referente a quantidade.\\
					\hline
					Capacidade Crianças & Onde deve ser informado um numero inteiro \\& referente a quantidade.\\
					\hline
					\end{tabular}
				\end{table}
				Ao Editar uma Acomodacoes o usuário deve informar o Código da Categoria a ser editada, que pode ser visto no menu Ler,ao inserir um Código valido(considera-se valido um código já cadastrado) e confirma sua ação,feito isso é preciso preencher novamente todas as informações presentes no menu criar, que são elas (Nome, Descrição, Valor Diaria, Capacidade de Adulto, Capacidade de Crianças).\\

				Ao Apagar uma Categoria o usuário deve informar o Código da Categoria a ser apagada,e confirmar sua ação. \\




				\newpage
				\subsection{Produtos}
				Para Criar um Produto é preciso ja ter um hotel cadastrado e informar alguns dados essenciais sendo eles:
				\begin{table}[h]%%Tabela
					\begin{tabular}{|c|c|}
					\hline
					Estoque  & Onde deve ser informado um numero Natural referente a \\& quantidade atual do produto em estoque.\\
					\hline
					Estoque Minimo  & Onde deve ser informado um numero  Natural  referente a \\& quantidade minima que se pode ter em estoque.\\
					\hline
					Descrição  & Onde deve ser informado qualquer tipo de caractere exceto \\& o [espaço] com no maximo 999 caracteres.\\
					\hline
					Preço Custo  & Onde deve ser informado um numero racional com 2 \\& casas depois da virgula referente ao valor unítario de compra.\\
					\hline
					Preço Venda  & Onde deve ser informado um numero racional com 2 \\& casas depois da virgula referente ao valor unítario de venda.\\
					\hline
					Código Hotel  & Onde deve ser informado um numero inteiro referente ao \\& hotel em que o produto está vinculado.\\
					\hline
					\end{tabular}
				\end{table}

				Ao Editar um Produto o usuário deve informar o Código do Produto a ser editado, que pode ser visto no menu Ler,ao inserir um Código valido(considera-se valido um código já cadastrado) e confirma sua ação,feito isso é preciso preencher novamente todas as informações presentes no menu criar, que são elas (Estoque, Estoque Minimo, Descrição, Preço Custo, Preço Venda, Código Hotel).

				Ao Apagar um Produto o usuário deve informar o Código do Produto a ser apagado,e confirmar sua ação. 


				\newpage
				\subsection{Fornecedores}
				\begin{table}[h]%%Tabela
					\begin{tabular}{|c|c|}
					\hline
					Nome Fantasia  & Onde deve ser informado qualquer tipo de caractere exceto \\& o [espaço] com no maximo 80 caracteres.\\
					\hline
					Razão Social  & Onde deve ser informado qualquer tipo de caractere exceto \\& o [espaço] com no maximo 80 caracteres.\\
					\hline
					Inscricao Estadual  & Onde deve ser informado qualquer tipo de caractere exceto \\& o [espaço] com no maximo 80 caracteres.\\
					\hline
					CNPJ & Onde deve ser informado qualquer tipo de caractere exceto \\& o [espaço] com no maximo 80 caracteres.\\
					\hline
					Logradouro & Onde deve ser informado qualquer tipo de caractere exceto \\& o [espaço] com no maximo 80 caracteres.\\
					\hline					
					Numero & Onde deve ser informado qualquer tipo de caractere exceto \\& o [espaço] com no maximo 80 caracteres.\\
					\hline					
					Bairro & Onde deve ser informado qualquer tipo de caractere exceto \\& o [espaço] com no maximo 80 caracteres.\\
					\hline
					Cidade & Onde deve ser informado qualquer tipo de caractere exceto \\& o [espaço] com no maximo 80 caracteres.\\
					\hline
					Telefone & Onde deve ser informado qualquer tipo de caractere exceto \\& o [espaço] com no maximo 80 caracteres.\\
					\hline
					Email & Onde deve ser informado qualquer tipo de caractere exceto \\& o [espaço] com no maximo 80 caracteres.\\
					\hline
					\end{tabular}
				\end{table}

				Ao Editar um Fornecedor o usuário deve informar o Código do Fornecedor a ser editado, que pode ser visto no menu Ler,ao inserir um Código valido (considera-se valido um código já cadastrado) e confirma sua ação,feito isso é preciso preencher novamente todas as informações presentes no menu criar, que são elas (Nome Fantasia, Razão Socia, Inscricao Estadual, CNPJ, Logradouro, Numero, Bairro, Cidade, Telefone, Email).

				Ao Apagar um Fornecedor o usuário deve informar o Código do Fornecedor a ser apagado,e confirmar sua ação. 

				\newpage
				\subsection{Funcionários}
				\begin{table}[h]%%Tabela
					\begin{tabular}{|c|c|}
					\hline
					Nome  & Onde deve ser informado qualquer tipo de \\& exceto o [espaço] caractere com no maximo 80 caracteres.\\
					\hline
					Usuario  & Onde deve ser informado qualquer tipo \\& caractere exceto o [espaço] de com no maximo 80 caracteres.\\
					\hline
					Senha  & Onde deve ser informado qualquer tipo \\& caractere exceto o [espaço] de com no maximo 80 caracteres.\\
					\hline
					Permissões &  					Onde deve ser digitado 1 para cada permissão concedida e 0 para  negada\\& 					no total são 4 permissoes [ Ler, Criar, Editar, Apagar] )(Nos arquivos .xml\\& 					 o conjunto das 4 permissoes formará em binário um valor entre 0 e 15).\\ 					
					\hline
					\end{tabular}
				\end{table}

				Ao Editar um Funcionário o usuário deve informar o Código do Funcionário a ser editado, que pode ser visto no menu Ler,ao inserir um Código valido (considera-se valido um código já cadastrado) e confirma sua ação,feito isso é preciso preencher novamente todas as informações presentes no menu criar, que são elas (Nome Fantasia, Razão Socia, Inscricao Estadual, CNPJ, Logradouro, Numero, Bairro, Cidade, Telefone, Email).\\

				Ao Apagar um Funcionário o usuário deve informar o Código do Funcionário a ser apagado,e confirmar sua ação. \\

				\newpage
				\section{Reserva}
				O modo de reserva serve para realizar uma reserva em uma acomodação no nome de algum hospede já cadastrado cadastrado, porém so é possivel fazer a reserva dentro de um més ou seja não é possivel reservar durante um periodo que exceda um mês.\\
				Tambem não é possivel realizar um reserva na mesma acomodação em que haja sombreamento de datas ou seja em uma acomodação que esteja ocupada em parte ou em seu todo durante certo periodo.\\
				Ao entrar no modo de reserva é mostrado para o usuário 3 opçoes [ler,criar,cancelar].\\

				\subsection{Ler Reserva}
				Esta opção mostra para o usuario todas as reservas registradas mesmo que o periodo de reserva ja tenha expirado.\\
				Mostrando os seguintes dados:

				\begin{itemize}
					\item Codigo
					\item Nome Hospede
					\item Codigo Acomodação
					\item Data Entrada[Dia/Mes/Ano]
					\item Data de vencimento da Fatura
					\item Valor da Fatura
					\item Pagamento
					\item Valor da Conta
					\item Modo de Pagamento
				\end{itemize}

				\newpage
				\subsection{Criar Reserva}
				Esta opção irá registrar uma nova reserva.\\
				Para que a reserva seja efetuada será pedido que informe os seguintes dados:
				\begin{table}[h]%%Tabela
					\begin{tabular}{|c|c|}
					\hline
					Codigo do Hospede\\
					\hline
					Codigo da Acomodação\\
					\hline
					Data de Entrada(Dia)(Mês)(Ano)\\
					\hline
					Data de Saida(Dia)\\
					\hline
					Data Vencimento da fatura(Dia)(Mês)(Ano)\\
					\hline
					Valor da Fatura\\
					\hline
					Situação da Fatura\\(1 para pago 0 para pendente)\\
					\hline
					Forma de Pagamento da reserva\\
					\hline
					Codigo do Produto Consumido\\
					\hline
					Quantidade do produto referente ao codigo anterior\\
					\hline
					Forma de Pagamento dos produto\\
					\hline
					\end{tabular}
				\end{table}
				\subsection{Cancelar Reserva}
				Para cancelar uma reserva é necessário que o yusuário informe o código da reserva que será cancelada.\\
				Ao informar um codigo válido (um codigo de uma reserva já cadastrada) é pedido para que confirme digitando \textbf{1} e pressionando enter.

				\newpage
				\section{Pesquisa}
				Para realizar uma pesquisa o usuário deve fazer um switch de opções a serem pesquisadas ( switch está relacionando ao fato de digitar 1 para cada opção que queira pesquisar e 0 para aquelas que não deseja realizar).\\
				\begin{table}[h]%%Tabela
					\begin{tabular}{|c|c|}
					\hline
					Data&Para pesquisar por data é necessário que seja informada uma data \\&referente a entrada (levando em consideração que o ano não pode\\& ser inferior ao ano de 2017) e o dia de saida\\&não pode ser  anterior ao dia de entrada.\\
					\hline
					Codigo Categoria&A pesquisa por codigo categoria  mostra uma lista com todas\\& as categorias já cadastradas, caso seja digitada um\\& codigo que não esteja cadastrado(na tela)\\& será pedido que re-informe o codigo.\\
					\hline
					Quantida de Pessoas&A pesquisa por quantidade de pessoas pede para o usuário \\& informar a quantidade de adultos que não pode ser inferior a 1  \\& e a quantidade de crianças.\\
					\hline
					Facilidade&Para realizar uma pesquisa por facilidades  o\\& usuário deve fazer um switch de opções a serem pesquisadas\\&(1 para pesquisar e 0 para não pesquisar).\\
					\hline
					\end{tabular}
				\end{table}


				\newpage
				\section{Importação/Exportação XML}
				A importação serve para fazer a transferencia de dados no formato de .xml porêm há outras formas de importar/exportar (copiando a pasta Arquivos).\\
				\subsection{Importação}
				Para Importar é necessário informar a localização do arquivo xml porem quando importamos todo os dados importados sobrescrevem os antigos.\\
				\subsection{Exportação}
				para exportar é necessário fazer um switch de quais  classe de dados seram exportadas.\\Ao digitar 1 significar que tal classe será exportada e ao digitar 0 não será 








	\end{document}
